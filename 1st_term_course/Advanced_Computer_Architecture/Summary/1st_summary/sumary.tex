\documentclass[10pt]{article}
\usepackage{NotesTeX}
%\usepackage{lipsum}

\title{{\Huge Summary}\\{\Large{DianNao: A Small-Footprint
High-Throughput Accelerator for Ubiquitous Machine-Learning
\footnote{Tianshi Chen. DianNao: A Small-Footprint
High-Throughput Accelerator for Ubiquitous Machine-Learning.
ASPLOS, 2014.}}}}
\author{Cui Weihao}
    
\affiliation{SJTU}
\emailAdd{reallygoodhao@126.com}
\begin{document}
    \maketitle
    \flushbottom
    \newpage
    \pagestyle{fancynotes}
    \part{Brief Introduction}
    In the paper, a novel machine-learning accelerator is designed for
    large-scale CNNs and DNNs, with a special emphasis on the impact
    of memory on accelerator design, performance and energy.
    \section{Main Contributions}
    \begin{itemize}
        \item A synthesized (place \& route) accelerator design for
        large-scale CNNs and DNNs, the state-of-the-art
        machine-learning algorithms.
        \item The accelerator achieves high throughput in a small area,
        power and energy footprint.
        \item The accelerator design focuses on memory behavior, and
        measurements are not cicumscribed to computational tasks, they
        factor in the performance and energy impact of memory transfer.
    \end{itemize}
    \section{Composition of Accelerator}
    \begin{itemize}
        \item Storage: an input buffer for input neurons (NBin), an out
        put buffer for output neurons (NBout), a third buffer for
        synaptic weights (SB).
        \item Computations: Neural Functional Unit.
        \item Control of Accelerator: Control Logic and  layer code.
    \end{itemize}
    %\section{Exprimental Results}
    \newpage
    \pagestyle{fancynotes}
    \part{Details of the Accelerator's design}
    \section{Neural Functional Unit (NFU)}
    The spirit of the NFU is to reflect the decomposition of a layer,
    which is included in CNNs and DNNs, into computational blocks.
    \begin{itemize}
        \item Arithmetic operators

        The computations of each layer type can be decomposed in
        either 2 or 3 stages. for classifer layers: multiplication
    \end{itemize}
    \section{Storage: NBin, NBout and SB}
    \section{Control and code}
\end{document}