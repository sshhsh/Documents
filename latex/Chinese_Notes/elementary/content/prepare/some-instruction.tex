
\section{一些说明}
\label{sec:some-instruction}

\subsection{求和符号与求积符号}
\label{sec:sum-prod-symbol}

今后会经常遇到类似于$x_1+x_2+\cdots+x_n$这样的求和,每次都写成这样长不免浪费篇幅,也显得累赘,因为这求和中实际上只有两个要素,一是求和通项$x_i$,二是求和的下标范围$1$到$n$,因此把这个求和简记为$\sum\limits_{i=1}^nx_i$,即
\[ \sum_{i=1}^n x_i = x_1+x_2+\cdots+x_n \]
求和符号中的$x_i$就是求和的通项,通项中的$i$称为求和指标,而上下标$i=1$和$n$分别给出了求和指标的上下限,即求和范围。

\begin{example}
  二项式定理利用求和符号就可以写成
  \[ (a+b)^n = \sum_{i=0}^n C_n^i a^{n-i}b^i \]
  而等比数列的求和公式则可以写成(公比不为1)
  \[ \sum_{i=0}^{n-1}q^i = \frac{1-q^n}{1-q} \]
\end{example}

在有求和符号参与的运算过程中,常常对求和指标进行变换,比如把上式的求和顺序倒过来,写成
\[ x_n+x_{n-1}+\cdots+x_1+x_0 \]
这时利用求和符号可以写成$\sum\limits_{i=n}^1x_i$,意义不变,也可以写成$\sum\limits_{i=1}^nx_{n+1-i}$,仍然没有本质上的不同,因为当$i$从1到$n$取值时,$n+1-i$便依次依次取值$n,n-1,\ldots,2,1$。因此,在接触并使用求和符号的初始阶段,将它展开来验证相关等式是有益的,以后熟悉了就可以省去展开操作。

有时会从求和符号中拆分出若干项来,也会将一些项合并到其中去,例如
\begin{eqnarray*}
  \sum_{i=0}^nx_i & = & x_0 + \sum_{i=1}^n x_i \\
  \sum_{i=1}^n x_i + x_{n+1} & = & \sum_{i=1}^{n+1} x_i
\end{eqnarray*}

在运算时,有时也会变换求和范围,比如对于
\[ \sum_{i=1}^{n} x_i = x_1 + x_1 + \cdots x_n \]
可将其变换为$\sum\limits_{i=0}^{n-1}x_{i+1}$,同样也是因为当$i$从$0$到$n-1$遍历取值时,$i+1$便依次取值$1,2,\ldots,n$。在这种变换下,便有如下的运算
\begin{eqnarray*}
  & & (1+t)\sum_{i=0}^{n-1}t^i - t \sum_{i=0}^{n-2}t^i \\
  & = & \sum_{i=0}^{n-1}t^i + t\sum_{i=0}^{n-1}t^i - t\sum_{i=0}^{n-2}t^i \\
  & = & \sum_{i=0}^{n-1}t^i + \sum_{i=0}^{n-1}t^{i+1} -\sum_{i=0}^{n-2}t^{i+1} \\
  & = & \sum_{i=0}^{n-1}t^i + \sum_{i=1}^{n}t^i - \sum_{i=1}^{n-1}t^i \\
  & = & \sum_{i=0}^{n}t^i
\end{eqnarray*}

在此基础上,还会遇到多重求和,比如对于一个$n$行$m$列的数阵(称为矩阵)的所有元素进行求和,这个数阵的形式为
\[
  \begin{pmatrix}
    a_{11} & a_{12} & \cdots & a_{1m} \\
    a_{21} & a_{22} & \cdots & a_{2m} \\
    \vdots & \vdots & \cdots & \vdots \\
    a_{n1} & a _{n2} & \cdots & a_{nm}
  \end{pmatrix}
\]
要对它的所有元素进行求和,可以先把每一行的元素加起来,得到每一行的和,再把每一行的和加起来,便得到所有元素的和。第$i$行元素的和是$\sum_{j=0}^ma_{ij}$,于是所有元素的和便是
\[ \sum_{i=1}^n \sum_{j=1}^m a_{ij} \]
同样,也可以先将每一列的元素加起来,再把每一列的和进行相加,显然所得出的结果应与前面一样,所以
\begin{eqnarray*}
  \sum_{i=1}^n \sum_{j=1}^m a_{ij} & = & \sum_{i=1}^n(a_{i1}+a_{i2}+\cdots+a_{im}) \\
                                   & = & (a_{11}+\cdots+a_{1m})+(a_{21}+\cdots+a_{2m})+\cdots+(a_{n1}+\cdots+a_{nm}) \\
                                   & = & (a_{11}+\cdots+a_{n1})+(a_{12}+\cdots+a_{n2})+\cdots+(a_{1m}+\cdots+a_{nm}) \\
                                   & = & \sum_{j=1}^m(a_{1j}+a_{2j}+\cdots+a_{nj}) \\
  & = & \sum_{j=1}^m\sum_{i=1}^na_{ij}
\end{eqnarray*}
于是得到二重求和的一个重要性质:可以交换两个求和指标的求和顺序,但这需要注意的是这里的两个求和指标的范围是互相独立的,如果不是这样则不能交换求和顺序,例如下面这样的求和
\[ \sum_{i=0}^n\sum_{j=0}^i x_ix_j \]

类比于求和符号,也可以用连乘积符号$\prod$来表示多个元素的连乘积,例如
\[ \prod_{i=1}^n \frac{1}{2^i} = \frac{1}{2} \cdot \frac{1}{4} \cdots \frac{1}{2^n} \]
同样也有多重求积,求积符号与求和符号有着类似的运算性质,不再重复说明了。

有些情况下的求和,将各项从左到右按下标列出比较繁琐,反而不如直接指明求和指标所应满足的限制条件列出,例如,$(a+b+c)^n$的展开式中,每一项都具有形式$t_{i,j,k}a^ib^jc^k$,其中$i,j,k \geqslant 0$并且$i+j+k=n$,如若按二重求和的写法应写为
\[ (a+b+c)^n = \sum_{i=0}^n \sum_{j=0}^{n-i} t_{i,j,n-i-j} a^i b^j c^{n-i-j} \]
其意义就是,首先$i$可以从$0$取到$n$,但是当$i$确定后,$j$就只能取$0$到$n-i$之间的值,因为要保证$i+j \leqslant n$,在$i$和$j$都取定后,$k$就只能取$n-i-j$了,然后按这样的取法进行求和。

显然这样做的本质是,将所有满足$i+j+k=n$的三元非负数对$(i,j,k)$进行排序,以适应求和符号对指标范围的硬性要求,但是这样写出来的结果反而不如$i+j+k=n$这个限定条件来的直观,因此我们直接使用简单的记号
\[ \sum_{i,j,k \geqslant 0, \  i+j+k=n} t_{i,j,k} a^ib^jc^k \]
来表示这种形式的求和,同样,在其它类似情形下,只要将求和指标所满足的限定条件列在求和符号下方就行了,例如对不超过$n$的素数进行求和
\[ \sum_{p \leqslant n}p \]
这里没有在条件中指明$p$为素数,这需要在上下文中进行指明。

%%% Local Variables:
%%% mode: latex
%%% TeX-master: "../../elementary-math-note"
%%% End:
