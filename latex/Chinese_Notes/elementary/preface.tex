
\chapter*{序}
\thispagestyle{empty}

这份笔记开工于2016年4月6日,其实它在这之前几年就应该动笔的,高考后的十年间,除了囫囵吞枣的学了一些高等数学内容外,数学能力没有与年龄同步增长,常引为恨事,所以打算以写作这份笔记为契机,给自己一个继续学习的机会,以期常有所思,常有所得。虽然平时也有一些数学上的思考,但终究没有积累下来,没有形成自己的知识体系,这份笔记算是在这方面的一个尝试。

其实我不能算是这份笔记的作者,只能算是整理者,因为其中属于我自己原创的内容其实很少,就这份笔记所涉及的内容而言,那些定理和结论都是前人几百年前甚至早在公元前就研究透彻了,我辈若是能在没接触前人成果的情况下独立发现一些结论就已经是非常不容易了,何况我是在早已经接触过这些结论甚至还大体记得推导证明过程的基础上进行了重新推导而已。这份笔记主要内容的来源,基本上是参考文献所列出的那几本书籍,这基本上也是我这几年所读过的书,实际上没有哪一部是认真读完过的,还有极少数的内容是我自己在没接触过前人结论的情况下自己推导所得(例如伯努利信封问题,我一直称为错位排列问题),这些内容主要以例子的形式出现。

这份笔记的写作受到了前苏联数学家菲赫金哥尔茨所著《微积分学教程》的影响,这是我最推崇的一部巨著,书中取材之广泛,讨论之深度和广度超乎我的想象能力,我也从这部书中受益良多,这份笔记也在取材的广度和深度方面甚至内容的组织方面都受到它的影响。

这份笔记目前没有成书之日,也没有什么计划之类,受限于自己的数学能力和工作闲暇,不定期的更新而已。

我最佩服的几何学家是古希腊的 Apollonius,中文译作阿波罗尼奥斯,他生活在公元前的古希腊,他所著的《圆锥曲线论》将圆锥曲线的性质几乎一网打尽,以致于后人在长达两千年间没能在这个领域有所建树,直到笛卡尔坐标几何的创立,他所采用的还是纯几何理论,当然他也是在一些前人的研究成果上结合自己的研究写出了这部巨著,有时间会认真读一读这部书的部分内容。让我惊叹的是这竟然是在公元前的希腊完成的,古希腊的数学到底是有多发达,何以古希腊没能在我们今天所称四大文明古国之列呢。

致谢也是不能缺少的,首先要感谢的是本书参考文献的作者们,是他们让我接触到了这么多精彩的数学内容,当然还有一些书没有在参考文献之列(其实是我也想不起来是从哪儿看到的了),也一并感谢。还要感谢的是《计算机程序设计艺术》一书的作者 Donald E. Knuth,他开发的\TeX 排版系统以及由之发展而来的 \LaTeX 系统,使得我排版自己的书籍成为可能,还有编辑器 Emacs 的作者 Richard Stallman,这个编辑器所带来的强大的功能和编辑体验对我完成这份笔记功不可没。最后还要感谢我的夫人和女儿,女儿的降生给了我们这个家庭前所未有的欢乐,我对她的信心是在她的学生生涯,数学学科不至于成为她的拦路虎。夫人在照顾女儿上的付出才让我得以有精力来完成这份笔记。还有我的父母和哥哥。

最后需要说明的是,这份笔记作为一份知识总结,没有在问题的引入和背景介绍方面下多少功夫(这远不是目前的篇幅所能完成的,而且也受作者经验和水平限制),因此并不适合于初学者用于学习新知识。

这份笔记的最新文档可以在以下链接找到,并在这个网页上不定期更新:

\hyperref[pdf-download-link]{https://coding.net/u/zhcosin/p/math-notes-publish/git/blob/master/elementary-math-notes.pdf}

有任何问题,欢迎邮件讨论: zhcosin@163.com.

\vspace{1.5cm}

\hfill zhcosin \hspace{5em}

\hfill 2017-03-20 于成都华阳 \hspace{1.5em}


%%% Local Variables:
%%% mode: latex
%%% TeX-master: "elementary-math-note"
%%% End:
